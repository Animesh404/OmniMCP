\documentclass{article}
\usepackage{graphicx}
\usepackage{hyperref}
\usepackage{amsmath, amssymb}
\usepackage{listings}
\usepackage{xcolor}

\title{OmniMCP: A Framework for Self-Generating UI Understanding Through Spatial-Temporal Synthesis}
\author{Richard Abrich \\ OpenAdapt.AI}
\date{March 2025}

\begin{document}

\maketitle

\begin{abstract}
We present OmniMCP, a novel framework that enables large language models to develop comprehensive UI understanding through the synthesis of spatial and temporal features. The framework combines fine-grained UI segmentation with process graphs derived from human demonstrations to construct rich contextual representations of interface states and interaction patterns. Our approach introduces a self-generating semantic layer that bridges the gap between raw UI elements and task-specific interaction strategies. Through advanced prompt templates and synthetic validation techniques, we demonstrate robust understanding across diverse interface patterns.

% Our experimental results show [X]\% improvement in task completion rates across [Y] different UI patterns compared to baseline approaches. The framework demonstrates particular effectiveness in [specific use case], achieving [metric] performance. Key technical innovations include [innovation 1] and [innovation 2].

\end{abstract}

\section{Introduction}
User interface automation remains a significant challenge in artificial intelligence, particularly in developing systems that can generalize across diverse interfaces and adapt to varying contexts. While recent advances in computer vision and natural language processing have improved UI element detection, existing approaches often lack the ability to synthesize spatial understanding with temporal interaction patterns.

This paper introduces OmniMCP, a framework that addresses these limitations through two key mechanisms:
\begin{itemize}
    \item Real-time UI structure analysis via OmniParser
    \item Temporal pattern learning through process graph representations
\end{itemize}

\section{Related Work}
The challenge of UI automation has been approached from multiple angles in recent years. Screen parsing approaches have focused on hierarchical element detection, while demonstration-based methods have emphasized pattern recognition. However, few approaches have attempted to synthesize both spatial and temporal understanding in a unified framework.

\section{Methodology}

\subsection{Framework Overview}
OmniMCP's architecture enables language models to generate semantic understanding by analyzing:
\begin{itemize}
    \item UI element hierarchies and spatial relationships
    \item Historical demonstration patterns encoded in process graphs
    \item Contextual mappings between current states and successful interaction sequences
\end{itemize}

\subsection{Core Components}
The framework consists of four tightly integrated components:

\begin{itemize}
    \item \textbf{Visual State Manager}: Handles UI element detection and state tracking
    \item \textbf{MCP Tools}: Provides typed interfaces for model-UI interaction
    \item \textbf{UI Parser}: Performs element detection and relationship analysis
    \item \textbf{Input Controller}: Manages precise interaction execution
\end{itemize}

The core data structures are defined as:

\begin{lstlisting}[language=Python]
@dataclass
class UIElement:
    type: str          # Element type (button, text, etc)
    content: str       # Semantic content
    bounds: Bounds     # Coordinates  
    confidence: float  # Detection confidence

@dataclass
class ScreenState:
    elements: List[UIElement]
    dimensions: tuple[int, int]
    timestamp: float
\end{lstlisting}

\subsection{Process Graph Representation}
We formalize UI automation sequences as directed graphs G(V, E) where vertices V represent UI states and edges E represent transitions through interactions. Each vertex contains:

\begin{itemize}
    \item Screen state representation S
    \item Element hierarchy H
    \item Interaction affordances A
\end{itemize}

Edges capture:
\begin{itemize}
    \item Interaction type T (click, type, etc.)
    \item Pre/post conditions P
    \item Success verification criteria V
\end{itemize}

This representation enables:
\begin{lstlisting}[language=Python]
class ProcessGraph:
    def validate_sequence(
        actions: List[Action]
    ) -> ValidationResult:
        """Validate action sequence against graph"""
    
    def suggest_next_actions(
        current_state: ScreenState
    ) -> List[Action]:
        """Suggest valid next actions"""
\end{lstlisting}

\subsection{Spatial-Temporal Feature Synthesis}
The core innovation lies in the dynamic synthesis of spatial and temporal features through our MCP protocol:

\begin{lstlisting}[language=Python]
@mcp.tool()
async def get_screen_state() -> ScreenState:
    """Get current state of visible UI elements"""
    state = await visual_state.capture()
    return state

@mcp.tool()
async def find_element(description: str) -> Optional[UIElement]:
    """Find UI element matching natural language description"""
    state = await get_screen_state()
    return semantic_element_search(state.elements, description)
\end{lstlisting}

\section{Evaluation}
\subsection{Benchmark Results}
\begin{table}[h]
\caption{Performance Comparison}
\begin{tabular}{|c|c|c|c|}
\hline
Metric & OmniMCP & Baseline 1 & Baseline 2 \\
\hline
Task Success Rate & & & \\
Completion Time & & & \\
Error Rate & & & \\
\hline
\end{tabular}
\end{table}

\subsection{System Architecture}
% TODO: Add figure
\begin{figure}[h]
\centering
% \includegraphics[width=0.8\textwidth]{system_architecture.png}
\caption{OmniMCP System Architecture}
\label{fig:architecture}
\end{figure}

\subsection{Complex UI Interaction Examples}
% TODO: Add example scenarios and results

\subsection{Spatial-Temporal Synthesis Algorithm}
%\begin{algorithm}
%\caption{Spatial-Temporal Feature Synthesis}
%\begin{algorithmic}
% TODO: Add algorithm steps
%\end{algorithmic}
%\end{algorithm}

\subsection{Process Graph Construction Example}
\begin{lstlisting}[language=Python]
def construct_process_graph(demonstrations: List[Demonstration]) -> ProcessGraph:
    # TODO: Implement process graph construction
    pass

def handle_edge_cases(graph: ProcessGraph) -> ProcessGraph:
    # TODO: Implement edge case handling
    pass
\end{lstlisting}

\subsection{Performance Optimizations}
Critical optimizations focus on maintaining reliable UI understanding:

\begin{itemize}
    \item \textbf{Minimal State Updates}: Update visual state only when needed, using smart caching and incremental updates
    \item \textbf{Efficient Element Targeting}: Optimize element search with early termination and result caching
    \item \textbf{Action Verification}: Verify all UI interactions with robust success criteria
    \item \textbf{Error Recovery}: Implement systematic error handling with rich context and recovery strategies
\end{itemize}

\section{Implementation}
The framework exposes a clean API for model interaction:

\begin{lstlisting}[language=Python]
async def describe_element(description: str) -> str:
    """Get rich description of UI element"""

async def find_elements(query: str) -> List[UIElement]:
    """Find elements matching natural query"""

async def click_element(
    description: str,
    click_type: str = "single"
) -> InteractionResult:
    """Click UI element matching description"""
\end{lstlisting}

\section{Model Context Protocol Implementation}
The framework implements a focused set of MCP tools for UI understanding:

\begin{lstlisting}[language=Python]
@mcp.tool()
async def get_screen_state() -> ScreenState:
    """Get current state of visible UI elements"""

@mcp.tool()
async def find_element(description: str) -> Optional[UIElement]:
    """Find UI element matching description"""

@mcp.tool()
async def click_element(description: str) -> ClickResult:
    """Click UI element matching description"""
\end{lstlisting}

This minimal but complete protocol enables:
\begin{itemize}
    \item Natural language interface
    \item Stateful context management
    \item Action verification
    \item Rich error handling
\end{itemize}

\section{Error Handling and Recovery}
The framework implements systematic error handling:

\begin{lstlisting}[language=Python]
@dataclass
class ToolError:
    message: str
    visual_context: Optional[bytes]  # Screenshot
    attempted_action: str
    element_description: str
    recovery_suggestions: List[str]
\end{lstlisting}

Key aspects include:
\begin{itemize}
    \item Rich error context
    \item Visual state snapshots
    \item Recovery strategies
    \item Debug support
\end{itemize}

\section{Synthetic UI Testing Framework}
We introduce a comprehensive synthetic testing framework that enables systematic validation without relying on real UIs:

\begin{lstlisting}[language=Python]
def generate_test_ui() -> Tuple[Image, List[Element]]:
    """Generate synthetic UI with known elements"""
    img = Image.new('RGB', (800, 600))
    elements = []
    # Draw UI elements with known positions
    draw.rectangle([(100, 100), (200, 150)], fill='blue')
    elements.append({
        "type": "button",
        "content": "Submit",
        "bounds": {"x": 100, "y": 100}
    })
    return img, elements

def generate_action_test_pair() -> Tuple[Image, Image, List[Element]]:
    """Generate before/after UI pairs for testing"""
    before_img, elements = generate_test_ui()
    after_img = simulate_action(before_img)
    return before_img, after_img, elements
\end{lstlisting}

This framework enables:
\begin{itemize}
    \item Platform-independent testing
    \item Deterministic validation
    \item Systematic scenario coverage
    \item CI/CD integration
\end{itemize}

The framework provides rich debugging context:

\begin{lstlisting}[language=Python]
@dataclass
class DebugContext:
    tool_name: str         # Operation performed
    inputs: Dict[str, Any] # Input parameters
    result: Any           # Operation result
    duration: float       # Execution time
    visual_state: Optional[ScreenState]
\end{lstlisting}

This enables systematic validation of the understanding synthesis process.

\section{Implementation Guidelines}
We provide structured guidelines for extending the framework:

\begin{itemize}
    \item \textbf{Core Principles}
    \begin{itemize}
        \item Visual state is always current
        \item Every action verifies completion
        \item Rich error context always available
        \item Debug information accessible
    \end{itemize}
    
    \item \textbf{Critical Functions}
    \begin{itemize}
        \item VisualState.update()
        \item MCPServer.observe()
        \item find\_element()
        \item verify\_action()
    \end{itemize}
    
    \item \textbf{Testing Requirements}
    \begin{itemize}
        \item Unit tests for core logic
        \item Integration tests for flows
        \item Visual verification
        \item Performance benchmarks
    \end{itemize}
\end{itemize}

\section{Configuration and Deployment}
The framework supports flexible deployment through:

\begin{itemize}
    \item Environment-based configuration
    \item Multiple parser deployment options
    \item Debug and logging controls
    \item Performance tuning parameters
\end{itemize}

\section{Limitations and Future Work}
Current limitations include:
\begin{itemize}
    \item Need for more extensive validation across UI patterns
    \item Optimization of pattern recognition in process graphs
    \item Refinement of spatial-temporal feature synthesis
\end{itemize}

Future work will focus on:
\begin{itemize}
    \item Development of comprehensive evaluation metrics
    \item Enhanced pattern recognition capabilities
    \item Expanded cross-platform validation
    \item Integration with broader LLM architectures
\end{itemize}

\section{Conclusion}
We present OmniMCP as a significant advance in self-generating UI understanding. Through the synthesis of spatial and temporal features, coupled with robust prompt engineering and systematic validation capabilities, our framework demonstrates strong potential for generalizable UI automation. While further validation is needed, initial results suggest OmniMCP represents a meaningful step toward more robust and adaptable UI understanding systems.

\section{References}
[TODO]
\end{document}